% LaTeX resume using res.cls
\documentclass[line,margin]{res} 
\usepackage{hyperref}
%\usepackage{helvetica} % uses helvetica postscript font (download helvetica.sty)
%\usepackage{newcent}   % uses new century schoolbook postscript font 

\setlength{\oddsidemargin}{-.5in}
\setlength{\evensidemargin}{-.5in}
\setlength{\textwidth}{6.0in}
%\setlength{\itemsep}{0in}
%\setlength{\parsep}{0in}

%\setlength{\topmargin}{-0.6in}  % Start text higher on the page
%\setlength{\textheight}{9.8in}  % increase textheight to fit more on a page
%\setlength{\headsep}{0.2in}     % space between header and text
%\setlength{\headheight}{12pt}   % make room for header

\usepackage{fancyhdr}  % use fancyhdr package to get 2-line header
\renewcommand{\headrulewidth}{0pt} % suppress line drawn by default by fancyhdr
\rfoot{Tian Cao  (\thepage/ 2)\\ \hspace*{-\sectionwidth}\emph{tiancao@cs.unc.edu} }  % put page number at right
\cfoot{}  % the footer is empty
\pagestyle{fancy} % set pagestyle for the document

\newenvironment{list1}{
  \begin{list}{\ding{113}}{%
      \setlength{\itemsep}{0in}
      \setlength{\parsep}{0in} \setlength{\parskip}{0in}
      \setlength{\topsep}{0in} \setlength{\partopsep}{0in} 
      \setlength{\leftmargin}{0.17in}}}{\end{list}}
\newenvironment{list2}{
  \begin{list}{$\bullet$}{%
      \setlength{\itemsep}{0in}
      \setlength{\parsep}{0in} \setlength{\parskip}{0in}
      \setlength{\topsep}{0in} \setlength{\partopsep}{0in} 
      \setlength{\leftmargin}{0.2in}}}{\end{list}}
      
{\def\section*#1{}\bibliography{ref}}
%\nocite{*}

\begin{document}

\name{\huge{\sc Tian Cao} \vspace*{.1in}}
% \address used twice to have two lines of address
%\address{1985  Storm Lane, Troy, NY 12180}
%\address{(518) 273-0014 or (518) 272-6666}

\begin{resume}
\section{\sc{Contact Information}}
\vspace{.05in}
\begin{tabular}{@{}p{3in}p{3in}}
Department of Computer Science & {\it Phone:}  (919)-699-9542 \\            
Sitterson Hall, UNC-Chapel Hill & {\it E-mail:}  tiancao@cs.unc.edu \\
Chapel Hill, NC 27599-3175 USA&  \url{http://cs.unc.edu/~tiancao/} \\
\end{tabular}
 
\section{\sc{Research Interests}}       
\smallskip

Image Analysis, Computer Vision, Machine Learning

\section{\sc Education}
\smallskip

\textsc{University \textit{of} North Carolina \textit{at} Chapel Hill}, Chapel Hill, NC, USA \hfill{since 08/2010}\\
\vspace*{-.1in}
\begin{list1}
\item[] Ph.D. candidate in Computer Science %\hfill{GPA: 3.7/4.0}
\end{list1}
 %\smallskip
 \vspace*{-.1in}
\textsc{Sichuan University}, Chengdu, Sichuan, China \hfill {09/2007-05/2010}\\
\vspace*{-.1in}
\begin{list1}
\item[] Master of Science in Computer Science  %\hfill {GPA: 3.8/4.0}
\end{list1}
 %\smallskip
 \vspace*{-.1in}
\textsc{Sichuan University}, Chengdu, Sichuan, China \hfill{09/2003-05/2007}\\
 \vspace*{-.1in}
\begin{list1}
 \item[]  Bachelor of Engineering in Computer Science %\hfill {GPA: 3.7/4.0}
\end{list1}

 
\section{\sc Research Experience} 
\smallskip

\textbf{Registration for Correlative Microscopy using Image Analogies} \hfill      2011-2012 \\
\textit{ Research Assistant} at UNC Chapel Hill  \hfill {Advisor: Marc Niethammer}\\
Developed an image analogies based multi-modal registration algorithm, extended the traditional image analogies method with sparse representation model, and applied this algorithm to the registration of Correlative Microscopy images.
\smallskip \\
%\textbf{Deformable Registration with built-in Invariant to Affine Transformations} \hfill      2010-2011 \\
%\textit{ Research Assistant} at UNC Chapel Hill \hfill {Advisor: Marc Niethammer}\\
%Developed a deformable registration algorithm with built-in affine transformations, which can update the affine transformation at each deformable registration iteration, and
%applied this algorithm to the registration between fluorescence microscopy and TEM images.
%\smallskip \\
\textbf{Energy based Crowd Motion Analysis} \hfill      2009-2010 \\
\textit{Research Assistant} at SIAT/CUHK \hfill {Advisor: Yangsheng Xu}\\
Developed a energy based crowd motion analysis algorithm based on mutual information, and
applied this algorithm to detect the crowd abnormal behaviors.
\smallskip \\
\textbf{Super Resolution and Anisotropic Diffusion for Ultrasound Speckle Reduction} \hfill      2009 \\
\textit{Research Assistant} at SCU/SASET \hfill {Advisor: Dong C. Liu}\\
Developed a fast and robust super-resolution method for intima reconstruction in medical
ultrasound imaging, and applied anisotropic diffusion to reduce speckle with edge enhancement during the image reconstruction.
\smallskip \\
\textbf{Motion Estimation and Visualization for Cardiac Ultrasound} \hfill      2008-2009 \\
\textit{Research Assistant} at SCU/SASET \hfill {Advisor: Dong C. Liu}\\
Developed a novel method for motion estimation and visualization of cardiac ultrasound images. The motion vector fields are derived from an adaptive curve region based matching algorithm.

\section{\sc Publications}  

[1].\textbf{Tian Cao}, Christopher Zach, Marc Niethammer et al., ``Registration for Correlative Microscopy using Image Analogies'',  \emph{Fifth Workshop on Biomedical Image Registration} (accepted).\vspace{.1in}\\
\smallskip
[2].\textbf{Tian Cao}, Bo Wang, Dong C. Liu, ``Optimized GPU Framework of Semi-implicit AOS
Scheme Based Speckle Reducing Nonlinear Diffusion'',  \emph{proceedings of SPIE Medical Imaging, 2009, Vol. 7259, 2009}.\vspace{.1in}\\
\smallskip
[3].Bo Wang, \textbf{Tian Cao}, Yuguo Dai, Dong C. Liu, ``Ultrasound Speckle Reduction via Super Resolution and Nonlinear Diffusion'',  \emph{the 9th Asian Conference on Computer Vision (ACCV 2009), 2009}. \vspace{.1in}\\
\smallskip
[4].\textbf{Tian Cao}, Chaowei Tan, Dong C. Liu, ``Adaptive Curve Region based Motion Estimation and Motion Visualization of Cardiac Ultrasound Imaging'',  \emph{the 3rd International Conference on Bioinformatics and Biomedical Engineering (ICBBE 2009), Vol. 3, pp. 453-457, 2009}.\vspace{.1in}\\
\smallskip
[5].\textbf{Tian Cao}, Xinyu Wu, Jinnian Guo, Shiqi Yu, Yangsheng Xu, ``Abnormal Crowd Motion Analysis'', \emph{IEEE International Conference on Robotics and Biomimetics (ROBIO 2009), 2009}.

 
\section{\sc Honors \& Awards}
\smallskip

%{\renewcommand\baselinestretch{1.1}\selectfont

Guanghua Scholarship.  \hfill       2010\vspace{.1in}\\
Graduate Student Fellowship 2009 (top 5\%), Sichuan University.  \hfill       2009\vspace{.1in}\\
Excellent undergraduate Student (top 5\%), Sichuan University.  \hfill       2007\vspace{.1in}\\
Student Innovation Award (top 3\%), Sichuan University.  \hfill       2006\vspace{.1in}\\
1st prize of China Undergraduate Mathematical Contest in Modeling (CUMCM).  \hfill       2006

\section{\sc Professional Skills}
\smallskip

C/C++, Java, Python, Matlab %, Windows, Linux, MaxOS X

%\par}
\end{resume}

%\bibliography{ref}
%\bibliographystyle{plain}
\end{document}







